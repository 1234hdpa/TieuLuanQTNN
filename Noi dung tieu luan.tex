%Phần thiết đặt trang
\documentclass[14pt,a4paper]{article}
\usepackage{mathastext}
\usepackage[utf8]{vietnam}
\usepackage{amsfonts}
\usepackage{amsmath}
\usepackage{amssymb}
\usepackage{graphicx}
\usepackage[left=2cm,right=2cm,top=2cm,bottom=2cm]{geometry}
\setlength{\parindent}{0pt}
\usepackage{parskip}
\setlength{\parskip}{0.5em}

%Đánh dấu đầu dòng bằng chữ a) b) c)
\usepackage[shortlabels]{enumitem}

%Font chữ hoa văn cho các chữ cái viết hoa
\usepackage[mathscr]{euscript}

%Điều chỉnh kích cỡ chữ cho tài liệu
\usepackage{scrextend}
\changefontsizes[20pt]{14pt}

%Thêm phần header cho trang
\usepackage{fancyhdr}
\pagestyle{fancy}
\renewcommand{\subsectionmark}[1]{\markboth{#1}{}}
\fancyhf{}
\fancyhead[R]{\leftmark}
\fancyfoot[C]{Trang \thepage}
\fancypagestyle{plain}{%
    \fancyhf{}%
    \renewcommand{\headrulewidth}{0pt}%
}

%Thêm đường line dưới footer cho trang
\renewcommand{\footrulewidth}{0.4pt}

%Điều chỉnh nội dung bảng danh sách hình vẽ
\usepackage[titles]{tocloft}
\newlength{\mylen}
\renewcommand{\cftfigpresnum}{\figurename\enspace}
\renewcommand{\cftfigaftersnum}{:}
\settowidth{\mylen}{\cftfigpresnum\cftfigaftersnum}
\addtolength{\cftfignumwidth}{\mylen}


%Màu sắc cho bảng
\usepackage[table,xcdraw]{xcolor}

%Chuyển hình EPS sang dạng PDF tránh gây lỗi font
\usepackage{epstopdf}

%Cải thiện sắp xếp vị trí hình ảnh, dùng kèm với [H] trong phần option ảnh
\usepackage{float}

%Dùng dấu chấm thay dấu hai chấm trong caption ảnh
\usepackage[labelsep=period]{caption}

%Phần thiết kế khung code nhập liệu
\usepackage{listings}
\usepackage{color}

\definecolor{dkgreen}{rgb}{0,0.6,0}
\definecolor{gray}{rgb}{0.5,0.5,0.5}
\definecolor{mauve}{rgb}{0.58,0,0.82}

\lstset{frame=tb,
  language=Matlab,
  aboveskip=3mm,
  belowskip=3mm,
  showstringspaces=false,
  columns=flexible,
  basicstyle={\small\ttfamily},
  numbers=left,
  numberstyle=\small\color{gray},
  keywordstyle=\color{blue},
  commentstyle=\color{dkgreen},
  stringstyle=\color{mauve},
  breaklines=true,
  breakatwhitespace=true,
  tabsize=3
}

%Mở tính năng đánh dấu công thức trong đoạn văn
\makeatletter
\newcommand*{\inlineequation}[2][]{%
  \begingroup
    % Put \refstepcounter at the beginning, because
    % package `hyperref' sets the anchor here.
    \refstepcounter{equation}%
    \ifx\\#1\\%
    \else
      \label{#1}%
    \fi
    % prevent line breaks inside equation
    \relpenalty=10000 %
    \binoppenalty=10000 %
    \ensuremath{%
      % \displaystyle % larger fractions, ...
      #2%
    }%
    ~\@eqnnum
  \endgroup
}
\makeatother

\usepackage{hyperref}

%Phần chọn font chèn câu lệnh giữa đoạn
\newenvironment{code}{\ttfamily}{\par}
\DeclareTextFontCommand{\chuyencode}{\code}

%Thêm chấm vào tiêu đề các phần
\usepackage{titlesec}
\titlelabel{\thetitle.\quad}

%Thêm định dạng cho các nút và menu lệnh.
\usepackage{menukeys}

%Đánh số cho các ví dụ và bài tập
\usepackage[thref,thmmarks,standard,amsmath,hyperref]{ntheorem}
\theoremheaderfont{\bfseries}
\theorembodyfont{\normalfont}
\theoremseparator{:}
\renewtheorem{example}{Ví dụ}
%Đánh số công thức theo chương
\numberwithin{equation}{section}

%Tự ngắt dòng cho biểu thức toán dài dòng
\usepackage{mathpazo}
\usepackage[mathpazo]{flexisym}
\usepackage{breqn}

%Nội dung chính
\begin{document}
%Chỉnh loại tiêu đề chương thành I, II
\renewcommand\thesection{\Roman{section}}
\renewcommand\thesubsection{\arabic{subsection}}
\renewcommand\thesubsubsection{\thesubsection.\arabic{subsubsection}}


%Chương 1:
\section{Các kiến thức chuẩn bị}
\subsection{Mở đầu và ví dụ về phương trình vi ngẫu nhiên}
\subsubsection{Mở đầu}
Trước tiên, ta tìm các quá trình $X=(X_t,t\geq 0)$ có thể thoả mãn phương trình
\begin{equation}\label{eq:1.1}
	\dfrac{dX_t}{dt}=b(t,X_t)+\sigma(t,X_t)W_t
\end{equation}
trong đó $b(t,x)\in\mathbb{R},\sigma(t,x)\in\mathbb{R}$ và $W_t$ là quá trình Wiener 1-chiều. Khi đó $X_t$ cũng là lời giải của phương trình tích phân
\begin{equation}\label{eq:1.2}
	X_t=X_0+\int_0^tb(s,X_s)ds+\int_0^t\sigma(s,X_s)dB_s
\end{equation}
hay viết dưới dạng vi phân là
\begin{equation}\label{eq:1.3}
	dX_t=b(t,X_t)dt+\sigma(t,X_t)dB_t
\end{equation}
Như vậy từ \eqref{eq:1.1} sang \eqref{eq:1.3} ta đã thay thế một cách hình thức $W_t$ bởi $\dfrac{dB_t}{dt}$. Vấn đề đặt ra lúc này với một phương trình vi phân \eqref{eq:1.3} thì:
\begin{enumerate}[(1)]
	\item Với những điều kiện nào của các hệ số b và $\sigma$ thì tồn tại nghiệm và nghiệm đó là duy nhất?
	\item Ta sẽ giải phương trình đó như thế nào?
\end{enumerate}
Ta hãy xét câu hỏi 2 qua một số ví dụ khác, sau đó ở phần sau sẽ giải quyết vấn đề trong câu hỏi 1.
\subsubsection{Một số ví dụ về phương trình vi phân ngẫu nhiên}
\textbf{a. Mô hình tăng dân số đơn giản}
\begin{equation}\label{eq:1.1.a}
	\dfrac{dN_t}{dt}=\alpha_t.N_t \condition{$N(0)=N_0$}
\end{equation}
trong đó $N_t$ là số dân tại thời điểm $t,\alpha_t$ là tốc độ tăng dân số tại thời điểm t. Thông thường $\alpha_t$ không được xác định rõ ràng nhưng do chịu tác động của môi trường tự nhiên nên người ta thường viết dưới dạng $\alpha W_t$, với $W_t$ là một quá trình Wiener còn $\alpha$ là hằng số. Ngoài ra, khi kết hợp với tỉ lệ sinh và tử cho ta thêm một hằng số khác là $r_t$. Do đó ta được $\alpha_t=r_t+\alpha W_t$ và thay vào phương trình \eqref{eq:1.1.a} sẽ được:
\begin{equation*}
	dN_t=rN_tdt+\alpha N_tdB_t
\end{equation*}
trong đó cụm $b(t,x)=rx$, còn $\sigma(t,x)=\alpha x$ hay
\begin{equation*}
	\dfrac{dN_t}{N_t}=rdt+\alpha dB_t\Leftrightarrow\int_0^t\dfrac{dN_t}{N_t}=rt+\alpha B_t \condition{$B_0=0$}\tag{*}
\end{equation*}
Để tính tích phân vế trái, ta áp dụng công thức Itô cho hàm $g(t,x)=\ln(x),x>0$ và ta có:
\begin{dmath*}
	d(\ln N_t)=\dfrac{1}{N_t}dN_t+\dfrac{1}{2}(-\dfrac{1}{N_t^2})(dN_t)^2=\dfrac{dN_t}{N_t}-\dfrac{1}{2N_t^2}\alpha^2N_t^2dt=\dfrac{dN_t}{N_t}-\dfrac{1}{2}\alpha^2.dt
\end{dmath*}
Do đó
\begin{equation*}
	\dfrac{dN_t}{N_t}=d(\ln N_t)+\dfrac{1}{2}\alpha^2dt
\end{equation*}
Kết hợp với hệ thức (*) ta suy ra:
\begin{equation*}
	\ln\dfrac{N_t}{N_0}=(r-\dfrac{1}{2}\alpha^2)\alpha t+\alpha B_t
\end{equation*}
Hay:
\begin{equation*}
	N_t=N_0.exp[(r-\dfrac{\alpha^2}{2})t+\alpha B_t]
\end{equation*}
Đó là lời giải của mô hình dân số đã cho.\\
\textbf{b. Chuyển động Brown trên đường tròn đơn vị}\\
Ta chọn $X=B$, một chuyển động Brown 1 chiều và $g(t,x)=e^{ix}=(\cos{x},\sin{x})\in\mathbb{R}^2$ với $x\in\mathbb{R}$. Khi đó
\begin{equation*}
	Y\equiv g(t,X)=e^{iB}=(\cos{B},\sin{B})
\end{equation*} 
Vậy
\begin{equation*}
	Y=(Y_1,Y_2) \condition{với $Y_1=\cos{B},Y_2=\sin{B}$}
\end{equation*}
Theo công thức Itô:
\begin{equation*}
	dY_1=-\sin{B.dB}-\dfrac{1}{2}\cos{B}dt
\end{equation*}
\begin{equation*}
	dY_2=-\cos{B.dB}-\dfrac{1}{2}\sin{B}dt
\end{equation*}
Như vậy, quá trình 2 chiều $Y=(Y_1,Y_2)$ mà ta gọi là chuyển động Brown trên đường tròn đơn vị, $(Y_1^2+Y_2^2=1)$ là lời giải của hệ phương trình vi ngẫu nhiên
\begin{equation*}
	dY_1=-\dfrac{1}{2}Y_1dt-Y_2dB,
\end{equation*}
\begin{equation*}
	dY_2=-\dfrac{1}{2}Y_2dt+Y_1dB
\end{equation*}
hoặc dưới dạng ma trận
\begin{equation*}
	dY=-\dfrac{1}{2}Y+KYdB,
\end{equation*} 
trong đó
\begin{equation*}
	Y=(Y_1,Y_2)\condition{$K=\begin{pmatrix} 0 & -1 \\ 1 & 0 \end{pmatrix}$}.
\end{equation*}
\subsection{Điều kiện về sự tồn tại và duy nhất của nghiệm}
\subsubsection{Định lý về sự tồn tại và duy nhất lời giải}

Ta xét phương trình vi phân ngẫu nhiên trong các điều kiện sau:
\begin{enumerate}[(1)]
	\item Không gian xác suất $(\Omega,\mathscr{F},P)$ với $\sigma$ - trường $\{\mathscr{F}_t,i\in [0,T]\}$
	\item Quá trình Wiener $m$ - chiều $W(t)=\left\{W^{(1)}(t),W^{(2)},...,W^{(m)}(t) \right\}$ thích nghi với họ $\sigma$ - trường $\{\mathscr{F}_t,t\in [0,T] \}$ (điều đó có nghĩa là $W(0)=0;\forall t\in[0,T]$ đại lượng $W(t)\mathscr{F}_t$ - đo được và số gia $W(t+s)-W(t)$ khi $s\geq0$ không phụ thuộc vào $\sigma$ - trường $\mathscr{F}_t$).
	\item $\xi_0$ là vectơ ngẫu nhiên $\mathscr{F}_0$ - đo được và vectơ $\xi_0$ không phụ thuộc vào quá trình $W(t)$.
	\item Các hàm số $a(t,x)$ và $\delta(t,x)\left(t\in [0,T];x\in R^m \right)$ nhận các giá trị trong $R^m$ và $\mathscr{L}(R^m)$ tương ứng, với $\mathscr{L}(R^m)$ là tập các toán tự tuyến tính từ $R^m$.\\
\end{enumerate}

\textbf{Dạng của phương trình vi phân ngẫu nhiên:}\\
\begin{equation}\label{eq:c3e1}
	d\xi(t)=a(t,\xi(t))dt+\delta(t,\xi(t))dW(t)
\end{equation}
với điều kiện ban đầu: $\xi(0)=\xi_0$\\
Phương trình \eqref{eq:c3e1} có thể viết dưới dạng tích phân:
\begin{equation*}
	\xi(t)=\xi_0+\int_{0}^{t}a(s,\xi(s))ds+\int_{0}^{t}\delta(s,\xi(s))dW(s) \condition{$t\in[0,T]$}
\end{equation*}
trong đó $\xi(t)$ là quá trình ngẫu nhiên cần tìm. Nghiệm của phương trình \eqref{eq:c3e1} với điều kiện ban đầu $\xi(0)=\xi_0$ là quá trình ngẫu nhiên m - chiều $\xi(t),t\in[0,T]$ sao cho:\\
\begin{enumerate}[a)]
	\item Quá trình $\xi(t)$ đo được dần đối với họ $\sigma$ - trường $\{\mathscr{F}_t,t\in[0,T]\}$
	\item Mọi toạ độ của vectơ ngẫu nhiên $\left\{a(t,\xi(t)),t\in[0,T] \right\}$ khả tích tuyệt đối trên đoạn $[0,T]$ với xác suất 1.
	\item Mọi phần tử của ma trận ngẫu nhiên $\left\{\delta(t,\xi(t)),t\in[0,T] \right\}$ bình phương khả tích trên đoạn $[0,T]$ với xác suất 1.
	\item Quá trình $\xi(t)$ có vi phân ngẫu nhiên xác định bởi 
\begin{center}
	$d\xi(t)=a(t,\xi(t))dt+\delta(t,\xi(t))dW(t)$ và $\xi(0)=\xi_0$
\end{center}
\end{enumerate}
Chú ý rằng khi $\delta(t,x)=0$ thì phương trình \eqref{eq:c3e1} sẽ là phương trình vi phân thường, thoả điều kiện ngẫu nhiên ban đầu. Như vậy ta sẽ giải phương trình vi phân thường với mỗi $\omega$ xác định.\\
Ta nói rằng phương trình \eqref{eq:c3e1} có nghiệm duy nhất nếu đối với 2 nghiệm bất kỳ $\xi_1(t),\xi_2(t)$ ta có hệ thức:\\
\begin{equation*}
	P\left\{\sup_{0\leq t\leq T}|\xi_1(t)-\xi_2(t)|>0 \right\}=0
\end{equation*}
Ta làm quen với định lý tồn tại và duy nhất nghiệm.\\
\textbf{ĐỊNH LÝ 3.1.2}\\
Giả sử các hệ số của phương trình \eqref{eq:c3e1} thoả các điều kiện:
\begin{enumerate}[(1)]
	\item Với mọi $t\in[0,T],x\in R^m$ ta có:
\begin{equation*}
	|a(t,x)|^2+|\delta(t,x)|^2\leq k(1+|x|^2)
\end{equation*}
trong đó k là một hằng số; còn $|\delta(t,x)|^2=\sum\limits_{i,j=1}^m(\delta^{ij}(t,x))^2\delta^{ij}(t,x)$ là các phần tử của ma trận $\delta(t,x)$
	\item Với mọi $R>0$, tồn tại hằng số $c_R$ sao cho, với $|x|\leq R;y\leq R$ và $t\in[0,T]$ ta sẽ có:
\begin{equation*}
	|a(t,x)-a(t,y)|^2+|\delta(t,x)-\delta(t,y)|^2\leq c_R|x-y|^2
\end{equation*}
Khi đó sẽ tồn tại duy nhất nghiệm liên tục $\xi(t),t\in[0,T]$ của phương trình \eqref{eq:c3e1}. Chứng minh định lý trong quyển \textbf{Stochastic Differential Equations - An Introduction with Application} của B. Oksendal.
\end{enumerate}
Mặc khác ta thấy rằng nghiệm của phương trình \eqref{eq:c3e1} có thể xác định bằng phương pháp xấp xỉ liên tiếp như sau:\\
Giả sử các hệ số $a(t,x),\delta(t,x)$ thoả điều kiện (1) của định lý 3.1.1 và điều kiện (2*) tồn tại hằng số $c>0$ sao cho $\forall t\in[0,T],x,y\in R^m$
\begin{equation*}
	|a(t,x)-a(t,y)|^2+|\delta(t,x)-\delta(t,y)|^2\leq c|x-y|^2
\end{equation*}
Ngoài ra ta còn giả định rằng $M|\xi_0|^2<\infty$. Khi đó, đặt $\eta_0(t)=\xi_0$ và:
\begin{equation*}
	\eta_{n+1}(t)=\xi_0+\int_{0}^ta(s,\eta_n(s))ds+\int_0^t\delta(s,\eta_n(s))dW(s) \condition{n=1,2,...}
\end{equation*}
Sau khi bộ sung thêm những mệnh đề phụ trợ ta sẽ có được bất đẳng thức:
\begin{equation*}
	E\sup_{0\leq t\leq T}|\eta_{n+1}(t)-\eta_n(t)|^2\leq\frac{M^nT^n}{n!} \condition{n=0,1,2,...}
\end{equation*}
trong đó M là một hằng số nào đó.\\
Các bất đẳng thức trên cho ta kết luận được rằng chuỗi:
\begin{equation*}
	\eta_0(t)+\sum_{n=0}^\infty(\eta_{n+1}(t)-\eta_n(t))
\end{equation*}
hội tụ đều với xác suất 1, nó chính là nghiệm của phương trình \eqref{eq:c3e1} khi các hệ số của phương trình thoả các điều kiện (1), (2) và điều kiện ban đầu $M|\xi_0|^2<\infty$.\\
Nếu các điều kiện của định lý 3.1.2 được thoả và $M|\xi_0|^{2p}<\infty$ với một số p nguyên bất kỳ thì nghiệm của phương trình \eqref{eq:c3e1} với điều kiện ban đầu $\xi_p$ thoả điều kiện $E|\xi(t)|^{2p}\leq k_p(1+E|\xi_0|^{2p})$ với $t\in[0,T]$ và $E|\xi(t)-\xi_0|^{2p}\leq k'_p(1+E|\xi_0|^{2p})$ với $t\in[0,T]$, trong đó $k_p,k'_p$ là những hằng số chỉ phụ thuộc vào $p,k$ và $T$.\\
Trong các điều kiện của định lý 3.1.2, nghiệm $\xi(t),t\in[0,T]$ của phương trình \eqref{eq:c3e1} có tính chất Markov đối với họ $\sigma$ - trường ${\mathscr{F}_t,t\in[0,T]}$. Điều đó đồng nghĩa với mọi $0\leq s\leq t\leq R,G\in \mathscr{B},\mathscr{B}$ là $\sigma$ - trường các tập Borel của $R^m$, với xác suất 1 ta sẽ có hệ thức:
\begin{equation*}
	P\{\xi(t)\in G/\mathscr{F}_s\}=P\{\xi(t)\in G/\xi(\mathscr{F})\}
\end{equation*}
Như vậy quá trình ngẫu nhiên $\xi(t),t\in[0,T]$ là một hàm ngẫu nhiên Markov với phân phối ban đầu:
\begin{equation*}
	\mu(G)=P\{\xi_0\in G\},G\in \mathscr{B}
\end{equation*}
Xác suất chuyển $p(s,x,t,G)$ của hàm ngẫu nhiên Markov $\xi(t)$ xác định bởi công thức:
\begin{equation*}
	p(s,x,t,G)=P\{\xi_{sx}(t)\in G\},0\leq s\leq t\leq T \condition{$x\in R^m,G\in\mathscr{B}$}
\end{equation*}
trong đó $\xi_{sx}(t)$ là nghiệm của phương trình:
\begin{equation}\label{eq:c3e2}
	\xi_{sx}(t)=x+\int_{s}^{t}a(\tau,\xi_{sx}(\tau))d\tau+\int_{s}^{t}\delta(\tau,\xi_{sx}(\tau))dW(\tau)
\end{equation}
Trong \eqref{eq:c3e2}, x là vectơ ngẫu nhiên trong $R^m,t\in[s,T]$\\
\textbf{Định lý 3.1.3}\\
Giả sử các hàm số $a(t,x)$ và $\delta(t,x)$ thoả các điều kiện của định lý 3.1.2, ngoài ra chúng là những hàm liên tục theo tập hợp các biến số. Khi đó quá trình ngẫu nhiên $\xi(t),t\in[0,T]$ là nghiệm của phương trình \eqref{eq:c3e1} sẽ là một quá trình khuếch tán với vectơ chuyển dịch $a(t,x)$ và ma trận khuếch tán:
\begin{equation*}
	B(t,x)=\delta(t,x)\delta^T(t,x)
\end{equation*}
Như vậy, lý thuyết về quá trình vi phân ngẫu nhiên cho ta khả năng xây dựng các quá trình khuếch tán trong những điều kiện khá rộng đối với các hệ số $a(t,x)$ và $B(t,x)$.\\
Ngoài ra, nếu bổ sung thêm về tính trơn của các hàm $a(t,x),\delta(t,x)$ ta có thể chứng minh sự tồn tại đạo hàm liên tục của hàm số:
\begin{equation*}
	u(s,x)=E\mathscr{f}(\xi_{sx}(t)),0\leq s\leq t\leq T,x\in R^m
\end{equation*}
theo biến x, khi đó ta sẽ thu được phương trình ngược Kolmogorov.\\
\textbf{Định lý 3.1.4}
Giả sử các hàm số $a(t,x)$ và $\delta(t,x)$ thoả điều kiện của định lý 3.1.2, chúng là những hàm liên tục và khả vi 2 lần theo biến $x$. Ngoài ra, với những số xác định $p>0$ và $k>0$ ta có bất đẳng thức:
\begin{dmath*}
	\sum\limits_{t,k=1}^{m}\left|\dfrac{\partial a^i(t,x)}{\partial x^k}\right|+\sum\limits_{i,j,k=1}^{m}\left|\dfrac{\partial^2a^i(t,x)}{\partial x^i\partial x^k}\right|+\sum_{i,j,k=1}^{m}\left|\dfrac{\partial\delta^{ij}(t,x)}{\partial x^k} \right|+\sum_{i,j,k,l=1}^{m}\left|\dfrac{\partial^2\delta^{ij}(t,x)}{\partial x^k\partial x^l} \right|\leq k(1+|x|^p)	
\end{dmath*}
khi đó nếu hàm $f(x),(x\in R^m)$ khả vi liên tục hai lần với các giá trị thực sao cho:
\begin{equation*}
	\left|f(x)\right|+\sum_{i=1}^{m}\left|\dfrac{\partial f(x)}{\partial x^2} \right|+\sum_{i,k=1}^{m}\left|\dfrac{\partial^2f(x)}{\partial x^i \partial x^k}\leq k(1+|x|^p \right| \condition{p>0}
\end{equation*}
ta sẽ có hàm:
\begin{equation*}
	u(s,x)=Ef(\xi_{sx}(t)),0\leq s\leq t \leq T,x\in R^m
\end{equation*}
trong đó $\xi_{sx}(t)$ là nghiệm của phương trình \eqref{eq:c3e2}, hai lần khả vi theo x, khả vi liên tục theo s và thoả phương trình:
\begin{equation*}
	\dfrac{\partial u(s,x)}{\partial s}+\sum_{i=1}^{m}a^i(s,x)\dfrac{\partial u(s,x)}{\partial x^2}+\dfrac{1}{2}\sum_{i,j,k=1}^{m}\delta^{ij}(s,x)\delta^{kj}(s,x)\dfrac{\partial^2u(s,x)}{\partial x^i\partial x^k}=0
\end{equation*}
trong miền $s\in(0,t),x\in R^m$ với điều kiện ban đầu là $\lim\limits_{s\rightarrow t}u(s,x)=f(x)$.\\
Hệ quả của định lý trên là sự tồn tại và duy nhất nghiệm của bài toán Cauchy đối với phương trình đạo hàm riêng dạng parabolic.\\
\textbf{Định lý 3.1.5}\\
Giả sử trong miền $0\leq s<T;x\in R^m$, xác định một toán tử vi phân
\begin{equation*}
	\mathscr{L}u(s,x)=\dfrac{\partial u}{\partial s}(s,x)+\sum_{i=1}^{m}a^i(s,x)\dfrac{\partial u(s,x)}{\partial x^i}+\dfrac{1}{2}\sum_{i,j=1}^{m}b^{ij}(s,x)\dfrac{\partial^2u(s,x)}{\partial x^i\partial x^j}
\end{equation*}
dạng parabolic. Nếu ma trận $B(t,x)$ với các phần tử $b^{ij}(t,x);i,j=1,2,...,m$ sao cho $B(t,x)=\delta(t,x)\delta^T(t,x)$ và các hàm $a(t,x),\delta(t,x)$ thoả điều kiện của định lý 3.1.4 thì khi đó bài toán Cauchy với:
\begin{equation*}
	\mathscr{L}u(s,x)=0,\lim\limits_{s\rightarrow T}u(s,x)=f(x)
\end{equation*} 
sẽ có nghiệm duy nhất với mọi hàm khả vi liên tục hai lần $f(x)$ đồng thời nó cùng với các đạo hàm riêng cho đến bậc hai của nó tăng đến vô cực không nhanh hơn một bậc nào bất kỳ của $|x|$. Trong trường hợp này, nghiệm $u(s,x)$ của bài toán Cauchy sẽ được biểu diễn dưới dạng:
\begin{equation*}
	u(s,x)=Ef(\xi_{sx}(t)),0\leq s<T,x\in R^m
\end{equation*}
với $\xi_{sx}(t),(t\in[s,T])$ là nghiệm của phương trình \eqref{eq:c3e2}\\
Định lý này chỉ ra rằng khi nghiên cứu về phương trình đạo hàm riêng dạng parabolic ta có thể sử dụng các kết quả của lý thuyết về phương trình vi phân ngẫu nhiên. Đặc biệt trong định lý 3.1.5 không đòi hỏi tính không suy biến của ma trận $B(t,x)$.\\
\textbf{Định nghĩa 3.1.6: Nghiệm mạnh và nghiệm yếu}\\
Nghiệm đã xác định trong định lý tồn tại và duy nhất 3.1.2 được gọi là một nghiệm mạnh vì các cơ sở sau đã được xác định trước
\begin{enumerate}[1)]
	\item Không gian xác suất cơ bản $(\Omega,\mathscr{F},P)$
	\item Họ các $\sigma$ - trường con đầy đủ của $\mathscr{F}:\{\mathscr{F}_t,t\in[0,1]\}$
	\item Quá trình Wiener $\{W_t,t\in[0,T]\}$ đã xác định sao cho $\{W_t,\mathscr{F}_t,t\in[0,T]\}$ lập thành Martingan.
\end{enumerate}
Khi chỉ cho trước các hàm $a(t,x)$ và $\delta(t,x)$ mà không cho trước ba yếu tố trên ta sẽ có nghiệm theo nghĩa yếu. Hay nói cách khác, phương trình:
\begin{equation*}
	d\xi_t=a(t,\xi)dt+\delta(t,\xi)dW(t)
\end{equation*}
được gọi là có nghiệm yếu nếu tìm được
\begin{enumerate}[(1*)]
	\item Không gian xác suất $(\Omega^*,\mathscr{F}^*,P^*)$
	\item Họ các $\sigma$ - trường con đầy đủ của $\mathscr{F}^*:\{\mathscr{F}^*_t,t\in[0,1]\}$
	\item Quá trình Wiener $\{W_t^*,t\in[0,T]\}$ sao cho $\{W_t^*,\mathscr{F}^8_t,t\in[0,T]\}$ lập thành Martingan.
	\item $\{\xi_t^*,\mathscr{F}_t^*,t\in[0,T]\}$ là một quá trình liên tục và thích nghi (tương thích) sao cho:
	\begin{equation*}
		d\xi_t^*=a(t,\xi)dt+\delta(t,\xi)dW(t)	
	\end{equation*}
\end{enumerate}
Như vậy ta nói rằng $\{\xi_t^*,t\in[0,T]\}$ là nghiệm yếu của phương trình vi phân ngẫu nhiên đã cho. Điều kiện ban đầu của nghiệm yếu là hàm phân phối xác suất $F$ cho trước, ta phải tìm nghiệm yếu sao cho $\xi_0^*$ có hàm phân phối bằng $F$.\\
\textbf{Chú ý:}
\begin{itemize}[*]
	\item Tính duy nhất của nghiệm mạnh được hiểu theo nghĩa có cùng quỹ đạo.
	\item Tính duy nhất của nghiệm yếu được hiểu theo nghĩa có cùng phân phối xác suất.
	\item Tên gọi nghiệm mạnh và nghiệm yếu được đặt ra vì nghiệm mạnh sẽ là nghiệm yếu, nhưng nghiệm yếu không thể trở thành nghiệm mạnh.
\end{itemize}
\textbf{2. Một số phương trình đặc biệt}\\
\textbf{Định nghĩa 3.2.1: Phương trình vi phân tuyến tính thuần nhất}\\
Phương trình vi phân ngẫu nhiên tuyến tính thuần nhất là phương trình có dạng:
\begin{equation}\label{eq:c3.2.1}
	d\xi(t)=\alpha(t)\xi(t)dt+\beta(t)\xi(t)dW_t
\end{equation} 
trong đó $\alpha(t),\beta(t)$ là các hàm số của t và $W_t$ là quá trình Wiener với điều kiện ban đầu $\xi(0)=\xi_0$\\
\textbf{Cách giải:} Trước tiên ta tìm phương pháp giải \eqref{eq:c3.2.1} trong trường hợp đặc biệt khi $\alpha(t)=0$. Cụ thể xét các ví dụ sau:\\
a) Giải phương trình: \inlineequation[eq:c3.2.2]{\begin{cases} d\xi(t)=\beta(t)\xi(t)dW_t \\ \xi(0)=1 \end{cases}}\\
Ta xét quá trình: \inlineequation[eq:c3.2.3]{\eta(t):=-\dfrac{1}{2}\int_{0}^{t}\beta^2(s)ds+\int_{0}^{t}\beta(s)dW_s} với vi phân tương ứng là: $d\eta=-\dfrac{1}{2}\beta^2(t)dt+\beta(t)dW_t$.\\
Sử dụng công thức Itô với $\varphi(t,x)=e^x$ ta sẽ có:
\begin{dmath*}
d\xi(t)=d\varphi(t,\eta_t)=\dfrac{\partial\varphi}{\partial x}d\eta+\dfrac{1}{2}\dfrac{\partial\varphi^2}{\partial x^2}\beta^2(t)dt=e^\eta(-\dfrac{1}{2}\beta^2(t)dt+\beta(t)dW_t+\dfrac{1}{2}\beta^2(t)dt)=\beta(t)\xi(t)dW_t
\end{dmath*}
Từ đó, theo \eqref{eq:c3.2.3} ta sẽ có nghiệm của phương trình \eqref{eq:c3.2.2} là:
\begin{equation*}
	\xi(t)=\exp\left\{-\dfrac{1}{2}\int_{0}^{t}\beta^2(s)ds+\int_{0}^{t}\beta(s)sW_s \right\}
\end{equation*}
Nếu ta thay điều kiện ban đầu $\xi(0)=1$ bởi điều kiện $\xi(0)=\xi_0$ ta sẽ có nghiệm của phương trình: \inlineequation[eq:c3.2.4]{\begin{cases}
	d\xi(t)=\beta(t)\xi(t)dW_t\\
	\xi(0)=\xi_0
\end{cases}} là quá trình ngẫu nhiên:
\begin{equation}\label{eq:c3.2.5}
	\xi(t)=\xi_0\exp\left\{\int_{0}^{t}\beta(s)dW_s-\dfrac{1}{2}\int_{0}^{t}\beta^2(s)ds \right\}
\end{equation}
b) Giải phương trình
\begin{equation}\label{eq:c3.2.6}
\begin{cases}
	d\xi(t)=\alpha(t)\xi(t)dt+\beta(t)\xi(t)dt\\
	\xi(0)=\xi_0
\end{cases}
\end{equation}
Ta tìm nghiệm của \eqref{eq:c3.2.6} dưới dạng: $\xi(t)=\xi_1(t).\xi_2(t)$ trong đó $\xi_1(t)$ thoả điều kiện: \inlineequation[eq:c3.2.7]{\begin{cases}
	d\xi_1(t)=\beta(t)\xi_1(t)dW_t\\
	\xi_1(0)=1
\end{cases}} và $\xi_2(t)$ thoả điều kiện: 
\begin{equation}\label{eq:c3.2.8}
\begin{cases}
	d\xi_2(t)=A(t)dt+B(t)dW_t\\
	\xi_2(0)=1
\end{cases}
\end{equation} với $A(t)$ và $B(t)$ là những hàm ta sẽ chọn sau này.\\
Khi đó ta có:
\begin{dmath*}
d\xi(t)=d(\xi_1,\xi_2)=\xi_1d\xi_2+\xi_2d\xi_1+\beta(t)\xi_1B(t)dt=\beta(t)\xi(t)dW_t+\xi_1d\xi_2+\beta(t)\xi_1B(t)dt	
\end{dmath*}
Ta chọn $A(t)$ và $B(t)$ sao cho:
\begin{equation*}
	d\xi_2(t)+\beta(t)B(t)dt=\alpha(t)\xi_2(t)dt
\end{equation*}
Cụ thể ta lấy: $A(t)=\alpha(t)\xi_2(t)$ và $B(t)=0$.\\
Từ hệ thức \eqref{eq:c3.2.8} sẽ cho ta:
\begin{equation}\label{eq:c3.2.9}
	\begin{cases}
		d\xi_2(t)=\alpha(t)\xi_2(t)dt\\
		\xi_2(0)=1
	\end{cases}
\end{equation}
Từ \eqref{eq:c3.2.9} sẽ cho ta:
\begin{equation*}
	\xi_2(t)=\exp\left(\int_{0}^{t}\alpha(s)ds \right)
\end{equation*}
Mặt khác theo phần a), phương trình \eqref{eq:c3.2.7} cho ta nghiệm là:
\begin{equation*}
	\xi_1(t)=\xi_0\exp\left\{\int_{0}^{t}\beta(s)dW_s-\dfrac{1}{2}\int_{0}^{t}\beta^2(s)ds \right\}
\end{equation*}
Kết hợp $\xi_1(t)$ và $\xi_2(t)$ ta có nghiệm của phương trình \eqref{eq:c3.2.6} là:
\begin{equation}\label{eq:c3.2.10}
	\xi(t)=\xi_1(t)\xi_2(t)=\exp\left\{\int_{0}^{t}\beta(s)dW_s+\int_{0}^{t}\left[\alpha(s)-\dfrac{1}{2}\beta^2(s) \right]ds\right\}
\end{equation}
\textbf{c) Phương trình Black-Scholes}\\
Phương trình mô tả sự biến động của giá cổ phiếu $S_t$ theo thời gian t:
\begin{equation}\label{eq:c3.2.11}
	dS_t=\alpha S_tdt+\beta S_tdW_t
\end{equation}
trong đó $\alpha$ và $\beta$ là những hằng số.\\
Đây là trường hợp đặt biệt của phương trình dạng \eqref{eq:c3.2.1}. Do đó từ \eqref{eq:c3.2.10} ta suy ra nghiệm của nó sẽ là:
\begin{equation}\label{eq:c3.2.12}
	S_t=\exp\left[\left(\alpha-\dfrac{\beta^2}{2} \right)t+\beta W_t \right]
\end{equation}
\textbf{d) Phương trình mũ Itô}\\
Dạng của phương trình như sau:
\begin{equation}\label{eq:c3.2.13}
	\begin{cases}
		dX(t)=\dfrac{1}{2}X(t)dt+X(t)dW_t\\
		X(0)=X_0
	\end{cases}
\end{equation}
Từ \eqref{eq:c3.2.10} ta suy ra nghiệm của nó sẽ là: $X(t)=X_0exp(W_t)$\\
\textbf{Định nghĩa 3.2.2: Phương trình vi phân tuyến tính dạng tổng quát}\\
Phương trình vi phân tuyến tính là phương trình có dạng
\begin{equation}\label{eq:c3.2.14}
	d\xi(t)=(\alpha(t)\xi(t)+f(t))dt+(\beta(t)\xi(t)+g(t))dW_t
\end{equation}
trong đó $\alpha(t),\beta(t),f(t),g(t)$ là những hàm của t và $W_t$ là quá trình Wiener với điều kiện ban đầu $\xi(0)=\xi_0$.\\
\textbf{Cách giải:} Ta tìm nghiệm của phương trình \eqref{eq:c3.2.14} dưới dạng: $\xi(t)=\xi_1(t)\xi_2(t)$, trong đó $\xi_1(t)$ là nghiệm của phương trình tuyến tính thuần nhất tương ứng.
\begin{equation}\label{eq:c3.2.15}
	\begin{cases}
		d\xi_1(t)=\alpha(t)\xi_1(t)dt+\beta(t)\xi_1(t)dW_t\\
		\xi_1(0)=1
	\end{cases}
\end{equation}
và $\xi_2(t)$ là nghiệm của phương trình:
\begin{equation*}
	\begin{cases}
		d\xi_2(t)=A(t)dt+B(t)dW_t\\
		\xi_2(0)=\xi_0
	\end{cases}
\end{equation*}
Hàm $A(t)$ và $B(t)$ ta sẽ chọn sau này. Khi đó ta có:
\begin{equation*}
	d\xi(t)=d(\xi_1(t)\xi_2(t))=\xi_1d\xi_2+\xi_2d\xi_1+\beta(t)\xi_1B(t)dt
\end{equation*}
Cụ thể ta chọn $A(t)$ và $B(t)$ sao cho:
\begin{equation*}
	\xi_1(t)[A(t)dt+B(t)dW_t]+\beta(t)\xi_1(t)B(t)dt=f(t)dt+g(t)dW_t
\end{equation*}
ta sẽ thu được: \inlineequation[eq:c3.2.16]{\begin{cases}
	A(t):=[f(t)-\beta(t)g(t)]\dfrac
	{1}{\xi_1(t)}\\
	B(t):=g(t)\dfrac{1}{\xi_1(t)}
\end{cases}}\\
Sử dụng kết quả ở phần b), theo công thức \eqref{eq:c3.2.14} ta có:
\begin{equation*}
	\xi_1(t)=\exp\left\{\int_{0}^{t}\beta(s)dW_s+\int_{0}^{t}\left[\alpha(s)-\dfrac{1}{2}\beta^2(s) \right]ds \right\}
\end{equation*}
Theo \eqref{eq:c3.2.16} ta có:
\begin{equation*}
	\xi_2(t)=\xi_0+\int_{0}^{t}[f(s)-\beta(s)g(s)](\xi_1(s))^{-1}ds+\int_{0}^{t}g(s)(\xi_1(s))^{-1}dW_s
\end{equation*}
Kết hợp $\xi_1(t)$ và $\xi_2(t)$ ta có nghiệm của phương trình \eqref{eq:c3.2.14} sẽ là:
\begin{equation*}
	\xi(t)=\xi_1(t)\xi_2(t)
\end{equation*}
\textbf{Định nghĩa 3.2.3: Phương trình Langevin}\\
Phương trình Langevin là phương trình có dạng:
\begin{equation}\label{eq:c3.2.17}
	\begin{cases}
		d\xi(t)=-b\xi(t)dt+\delta dW_t\\
		\xi(0)=\xi_0
	\end{cases}
\end{equation}
trong đó $b,\delta$ là những hằng số thực.\\
Phương trình Langevin là một phương trình tuyến tính, do đó áp dụng kết quả phần II, theo công thức \eqref{eq:c3.2.17} ta thu được nghiệm của nó là phương trình ngẫu nhiên:
\begin{equation}\label{eq:c3.2.18}
	\xi(t)=e^{-bt}\left[\xi_0+\int{0}^{t}\delta e^{bs}dW_s \right] \condition{$t\geq 0$}
\end{equation}
Khi đó quá trình xác định bởi \eqref{eq:c3.2.18} được gọi là quá trình Ornstein-Unlenbeck.\\
\textbf{Định nghĩa 3.2.4: Phương trình Langevin mở rộng}\\
Phương trình Langevin mở rộng là phương trình có dạng:
\begin{equation}\label{eq:c3.2.19}
	\begin{cases}
		d\xi(t)=\alpha(t)\xi(t)dt+\delta(t)dW_t\\
		\xi(0)=\xi_0
	\end{cases}
\end{equation}
trong đó $\alpha(t)$ là một quá trình liên tục thích nghi.\\
Từ lời giải của phương trình vi phân tuyến tính tổng quát ta sẽ có nghiệm của phương trình Langevin mở rộng là:
\begin{equation}\label{eq:c3.2.20}
	\xi(t)=\exp\left(\int_{0}^{t}\alpha(s)ds\right)\left[\xi_0+\int_{0}^{t}\exp\left(-\int_{0}^{r}\alpha(s)ds \right)\delta(r)dW_r \right]
\end{equation}
Xét phương trình tuyến tính có dạng:
\begin{equation}\label{eq:c3.2.21}
	\begin{cases}
		d\xi(t)=\alpha(t)\xi(t)+f(t)]dt+\sum\limits_{i=1}^{m}\left[\beta^i(t)\xi(t)+g^i(t)\right]dW_t^i\\
		\xi(0)=\xi_0
	\end{cases}
\end{equation}
Khi đó nghiệm của nó sẽ là:
\begin{dmath}\label{eq:c3.2.22}
\xi(t)=G(t)\left[\xi_0+\int_{0}^{t}\left(G(s)\right)^{-1}\left(\alpha(s)-\sum\beta^i(s)g^i(s)\right)ds \right]+\int_{0}^{t}\sum\limits_{i=1}^{m}(G(s))^{-1}g^i(s)dW_t^i	
\end{dmath}
trong đó:
\begin{equation*}
	G(t):=\exp\left(\int_{0}^{t}\left[\alpha(s)-\sum_{i=1}^{m}\dfrac{(\beta^i)^2}{2} \right]ds+\int_{0}^{t}\sum_{i=1}^{m}\beta^i(s)dW_s^i \right)
\end{equation*}

\textbf{Nội dung trong tập}\\
\textbf{* Giải tích cũ:} Nhận dạng $F(x,y,y',y'',...,y^{(n)})$ rồi giải phương trình \begin{equation}
	\begin{aligned}
		F(x,dx,d^2x,...,d^{(n)}x)=0
	\end{aligned}
\end{equation}
Trong đó nếu gặp dạng đặc biệt ta giải phương trình vi phân tuyến tính theo dạng thuần nhất rồi đến dạng không thuần nhất.\\
Xét: $a_ny^{(n)}_x+a_{n-1}y^{(n-1)}_x+...+a_1y'(x)+a_0y(x)=f(x)$ nếu $f(x)=0$ thì phương trình tuyến tính thuần nhất, ngược lại nếu $f(x)\neq0$ thì phương trình tuyến tính không thuần nhất.\\
* Phương pháp giải: Giải phương trình thuần nhất (tương ứng) theo phương pháp biến thiên hằng số tức là:
\begin{gather*}
	ay''+by'+cy=f\Rightarrow ay''+by'+cy=0
\end{gather*}
Khi đó phương trình đặc trưng $ak^2+bk+c=0$ có nghiệm tổng quát là:\\
$\begin{cases} y(x)=c_1e^{k_1x}+c_2e^{k_2x} & (k_1\neq k_2 \in \mathbb{R}) \\ y(x)=(c_1x+c_2)e^{kx} & (k_1=k_2=k\in \mathbb{R}) \\ y(x)=e^{\alpha x}(c_1\cos{\beta x}+c_2\sin{\beta x}) & (k_{1,2}=\alpha \pm \beta i) \end{cases}$\\
\textbf{* Giải tích ngẫu nhiên}: Giải phương trình $dX_t=\alpha(\omega,t)dt+\beta(\omega,t)dW_t$ hay dạng tổng quát $dX_t=\alpha(\omega,t)dt+\beta(\omega,t)d\aleph_t$ với $\aleph_t$ là quá trình có vi phân ngẫu nhiên.\\
* Phương pháp giải: Khi giải phương trình vi phân tuyến tính thuần nhất trước, sau đó ta dùng phương pháp biến đổi hằng số để trừ ra nghiệm tổng quát của phương trình vi phân tuyến tính không thuần nhất.\\
* Nhận dạng phương trình vi phân tuyến tính:\\
$dX_t=\alpha X_tdt+\beta X_tdW_t \rightarrow$ phương trình vi phân tuyến tính thuần nhất.\\
$dX_t=(\alpha X_t+F)dt+(\beta X_t+G)dW_t$ nếu $F\equiv G \equiv 0$ thì không thuần nhất.\\
* Phương pháp chung: Dùng công thức Itô $\rightarrow$ Giải phương trình thuần nhất $\Rightarrow$ Dùng phương pháp tách biến để giải phương trình không thuần nhất.\\
\textbf{* Phương trình Black-Scholes}: $dX=\alpha dt+\beta dW$\\
B1: Giải thuần nhất thiếu (khử $\alpha$) suy ra $X_1$.\\
B2: Dùng phương pháp tách nghiệm $X=X_1.X_2$ để tìm $X_2$.\\
B3: Trong $X_2$ có A, B chưa biết, tìm A, B.
B4: Có $X_2$ kết hợp với $X_1$ để có $X\Rightarrow$ Giải xong phương trình thuần nhất đủ.\\
B5: Giải phương trình không thuần nhất từ kết quả trước đó.\\
B6: Dùng phương pháp tách biến để thực hiện B5.\\
B7: Sau khi tách biến, kết quả giống với B1, lại tìm A, B chưa biết khác.\\
B8: Có $\xi_2$ lại kết hợp $\xi_1$ để ra $\xi$ là nghiệm cần tìm.

\textbf{Quyển "Lớp quá trình ngẫu nhiên Itô - Levy và ứng dụng"}\\
\textbf{1. Phương trình vi phân tuyến tính Itô - Levy}\\
a. Định nghĩa Phương trình vi phân tuyến tính Itô - Levy\\
\textbf{Định nghĩa:} Phương trình vi phân tuyến tính Itô - Levy là phương trình có dạng:
\begin{dmath}\label{eq:dn1}
	dX(t)=[\alpha(t,\omega)X(t^-)+A(t,\omega)]dt+[\beta(t,\omega)X(t^-)+B(t,\omega)]dW(t)\\
	+\int_{(R_0)^{n_2}}[\gamma(t,z,\omega)X(t^-)+G(t,z,\omega)]\overline{N}(dt,dz)
\end{dmath}
Trong đó điều kiện ban đầu $X(0)=x_0$, với:
\begin{center}
$\alpha(t,\omega); \beta(t,\omega);A(t,\omega);B(t,\omega);\gamma(t,z,\omega);G(t,z,\omega);$\\
$\forall t \geqslant 0; z \in R_0; \omega \in \Omega$
\end{center}
là những quá trình ngẫu nhiên khả đoán cho trước với:
\begin{center}
	$\gamma(t,z,\omega)>-1;\forall (t,z,\omega)\in [0,\inf\}\times R_0 \times \Omega$
\end{center}
và thoả các điều kiện:
\begin{center}
	$\int\limits_{0}^t[|\alpha(s,\omega)|-\dfrac{1}{2}\beta^2(s,\omega)+\int\limits_{R_0} \gamma^2(s,z,\omega)v(dz)]ds<\infty$; h.c\\
	$\int\limits_{0}^t [|A(s,\omega)-\dfrac{1}{2}B^2(s,\omega)+\int\limits_{R_0}G^2(s,z,\omega)v(dz)]ds<\infty$; h.c
\end{center}
khi $A(t,\omega)\equiv B(t,\omega)\equiv G(t,z,\omega)\equiv 0$, h.c; ta gọi đó là quá trình vi phân ngẫu nhiên tuyến tính thuần nhất hay còn gọi là phương trình vi phân ngẫu nhiên hình học.\\
b. Định lý Giải phương trình vi phân ngẫu nhiên\\
Cho phương trình vi phân ngẫu nhiên tuyến tính Itô - Levy nêu trong định nghĩa trên, khi đó phương trình \eqref{eq:dn1} sẽ có nghiệm là:
\begin{dmath}\label{eq:dl1}
	\dfrac{X(t)}{X_1(t^-)}=\left\{x_0+\int_{R_0}\dfrac{1}{X_1(s^-)}[A(s,\omega)-\beta(s,\omega)B(s,\omega)-\int_{R_0} \dfrac{\gamma(s,z,\omega)G(s,z,\omega)}{1+\gamma(s,z,\omega)}v(dz)]ds+\int_{0}^{t} \dfrac{B(s,\omega)}{X_1(s)}+\int_{0}^{t} \int_{R_0} \dfrac{G(s,z,\omega)}{X_1(s^-)(1+\gamma(s,z,\omega))} \overline{N}(ds,dz) \right\}
\end{dmath}
Trong đó:
\begin{dmath}\label{eq:pt1}
	X_1(t)=exp\left\{\int_{0}^{t}\left[\alpha(s,\omega)-\dfrac{1}{2}\beta^2(s,\omega)+\int_{R_0}\log(1+\gamma(s,z,\omega))-\gamma(s,z,\omega)v(dz) \right]ds+\int_{0}^{1}\beta(s,\omega)dW(s)+\int_{0}^{1}\int_{R_0}\log(1+\gamma(s,t,\omega))\overline{N}(ds,dz) \right\}
\end{dmath}
Để chứng minh định lý trên trước hết ta cần chứng minh bổ đề sau:\\
c. Bổ đề\\
Cho phương trình vi phân ngẫu nhiên tuyến tính thuần nhất, nghĩa là phương trình có dạng:
\begin{dmath}\label{eq:pt2}
	\dfrac{dX_1(t)}{X_1(t^-)}=\left[\alpha(t,\omega)dt+\beta(t,\omega)dW(t)+\int_{R_0}\gamma(t,z,\omega)\overline{N}(dt,dz) \right]	
\end{dmath}
Với điều kiện ban đầu $X(0)=1$, trong đó:
\begin{equation*}
\alpha(t,\omega);\beta(t,\omega);\gamma(t,z,\omega);t\geq0;z\in R_0;\omega\in \Omega	
\end{equation*}
là những quá trình ngẫu nhiên khả đoán cho trước với
\begin{equation*}
	\gamma(t,z,\omega)>-1;\forall (t,z,\omega)\in [0;\infty\}\times R_0 \times \Omega
\end{equation*}
và thoả điều kiện:
\begin{equation*}
	\int\limits_{0}^{t}\left[|\alpha(s,\omega)|-\dfrac{1}{2}\beta^2(s,\omega)+\int_{R_0}\gamma^2(s,z,\omega)v(dz) \right]ds<\infty \condition{h.c}
\end{equation*}
khi đó nghiệm của phương trình \eqref{eq:pt2} sẽ được cho bởi hệ thức \eqref{eq:pt1}.\\
\textbf{Chứng minh bổ đề}:\\
Ta xét hàm $X_1(t)=F(t,H(t));t\geq 0$ với $F(t,x)=e^x$ và $H(T)$ xác định bởi:
\begin{dmath*}
H(t)=\int\limits_{0}^{t}\left[\alpha(s,\omega)-\dfrac{1}{2}\beta^2(s,\omega)+\int_{R_0}\log(1+\gamma(s,z,\omega))-\gamma(s,z,\omega)v(dz) \right]ds+\int_{0}^{1}\beta(s,\omega)dW(s)+\int_{0}^{1}\int_{R_0}\log(1+\gamma(s,z,\omega))\overline{N}(ds,dz)	
\end{dmath*}
Áp dụng công thức Itô cho $X_1(t)=F(t,H(t))$, ta sẽ thu được:
\begin{dmath*}
dX_1(t)=e^{H(t)}\left[\left(\alpha(t,\omega)-\dfrac{1}{2}\beta^2(t,\omega)+\int_{R_0}[\log(1+\gamma(t,z,\omega))-\gamma(t,z,\omega)]v(dz) \right)dt\right]+e^{H(t)}\left[\dfrac{1}{2}\beta^2(t,\omega)dt+\beta(t,\omega)dW(t) \right]	+\int_{R_0}e^{H(t)}[\gamma((t,z,\omega)-\log(1+\gamma(t,z,\omega)))]v(dz)dt+\int_{R_0}e^{H(t^-)}\gamma(t,z,\omega)\tilde{N}(dt,dz)=X_1(t^-)\left[\alpha(t,\omega)dt+\beta(t,\omega)dW(t)+\int_{R_0}\gamma(t,z,\omega)\tilde{N}(dt,dz) \right]\blacksquare
\end{dmath*}
\textbf{Chứng minh định lý}\\
Ta tìm nghiệm phương trình \eqref{eq:dn1} bằng phương pháp tách nghiệm, nghĩa là tìm nghiệm của nó dưới dạng tích
\begin{equation}\label{eq:pt3}
	X(t)=X_1(t^-).X_2(t^-)
\end{equation}
trong đó:
\begin{itemize}
	\item $X_1(t)$ là nghiệm của phương trình tuyến tính thuần nhất tương ứng, nghĩa là nó là nghiệm của phương trình \eqref{eq:pt2} nói trong Bổ đề.
	\item $X_2(t)$ là nghiệm của phương trình:
\begin{equation*}
	dX_2(t)=A^*(t,\omega)dt+B^*(t,\omega)dW(t)+\int_{R_0}G^*(t,z,\omega)\tilde{N}(dt,dz)
\end{equation*}
\end{itemize}
Với điều kiện $X_2(0)=x_0$, trong đó $A^*(t,\omega);B^*(t,\omega);G^*(t,z,\omega)$ là những hàm ta sẽ xác định sau.\\
Theo bổ đề, ta có nghiệm $X_1(t^-)$ của phương trình \eqref{eq:pt2} cho bởi hệ thức \eqref{eq:pt1}. Áp dụng hệ quả cho tích $X(t)=X_1(t^-).X_2(t^-)$ nêu trên, ta thu được:
\begin{dmath}
d(X(t))=d(X_1(t^-).X_2(t^-))=X_1(t^-).dX_2(t)+X_2(t^-)dX_1(t)+\beta(t,\omega)X_1(t^-)B^*(t,\omega)dt+\int_{R_0}\gamma(t,z,\omega)X_1(t^-)G^*(t,z,\omega)\tilde{N}(dt,dz)=\alpha(t,\omega)X_1(t^-)X_2(t^-)+\beta(t,\omega)X_1(t^-)X_2(t^-)+\int_{R_0}\gamma(t,z,\omega)X_1(t^-)X_2(t^-)\tilde{N}(dt,dz)+X_1(t^-)A^*(t,\omega)dt+X_1(t^-)B^*(t,\omega)dW(t)+X_1(t^-)\int_{R}G^*(t,z,\omega)\tilde{N}(dt,dz)+\beta(t,\omega)X_1(t^-)B^*(t,\omega)dt+\gamma(t,z,\omega)X_1(t^-)G*(t,z,\omega)\tilde{N}(dt,dz)	
\end{dmath}
Mặt khác, $X(t)$ là nghiệm của phương trình \eqref{eq:dn1}, từ đó so sánh giữa \eqref{eq:dn1} và \eqref{eq:pt3}, ta thu được hệ phương trình:
\begin{dmath*}
\begin{cases}
	A(t,\omega)=X_1(t^-)\left[A^*(t,\omega)+B(t,\omega)B^*(t,\omega)+\int\limits_{R_0}\gamma(t,z,\omega)G(t,z,\omega)v(dz) \right] \\
	B(t,\omega)=X_1(t^-)B^*(t,\omega) \\
	\int\limits_{R_0}G(t,z,\omega)\tilde{N}(dt,dz)=X_1(t^-)\int\limits_{R_0}(1+\gamma(t,z,\omega))G^*(t,z,\omega)\tilde{N}(dt,dz)
\end{cases}	
\end{dmath*}
Suy ra:
\begin{dmath*}
\begin{cases}
	A^*(t,\omega)=\dfrac{1}{X_1(t^-)} \left[A(t,\omega)-B(t,\omega)\beta(t,\omega)-\int_{R_0} \dfrac{\gamma(t,z,\omega)G(t,z,\omega)}{1+\gamma(t,z,\omega)} v(dz) \right] \\
	B^*(t,\omega)=\dfrac{B(t,\omega)}{X_1(t^-)} \\
	G^*(t,z,\omega)=\dfrac{G(t,z,\omega)}{X_1(t^-)(1+\gamma(t,z,\omega)}
\end{cases}	
\end{dmath*}
Đặt $X_1(t)$ cho bởi hệ thức \eqref{eq:pt1}, và các biểu thức của $A^*(t,\omega);B^*(t,\omega);G^*(t,z,\omega)$ đã xác định được vào biểu thức \eqref{eq:pt3}, ta sẽ có nghiệm \eqref{eq:dn1}. $\blacksquare$\\
\textbf{2. Một số phương trình đặc biệt}\\
a. Phương trình Langevin\\
\textbf{Phương trình Langevin} là phương trình có dạng:
\begin{equation*}
	dX(t)=-bX(t)dt+\sigma dW_t \condition{$X(0)=0$}
\end{equation*}
trong đó $b, \sigma$ là những hằng số thực.\\
Phương trình Langevin là một phương trình vi phân ngẫu nhiên tuyến tính, do đó, theo kết quả phần trước ta thu được nghiệm của nó là quá trình ngẫu nhiên:
\begin{equation*}
	X(t)=e^{-bt}\left[X(0)+\int\limits_{0}^{t}\sigma e^{bs}dW_s \right] \condition{$t\geq 0$}
\end{equation*}
Quá trình xác định bởi hệ thức trên được gọi là quá trình Ornstein-Uhlenbeck.\\
\textbf{Phương trình Langevin mở rộng} là phương trình có dạng:
\begin{equation*}
	dX(t)=\alpha(t)X(t)dt+\sigma (t)dW_t \condition{$X(0)=X_0$}
\end{equation*}
trong đó $\alpha(t)$ và $\sigma(t)$ là những quá trình ngẫu nhiên liên tục.\\
Từ lời giải của phương trình vi phân tuyến tính tổng quát, ta sẽ có nghiệm của phương trình Langevin mở rộng là:
\begin{equation*}
	X(t)=\exp\left(\int\limits_{0}^{t}\alpha(s)ds \right)\left[X_0+\int\limits_{0}^{t}\exp\left(-\int\limits_{0}^{r}\alpha(s)ds \right)\sigma(r)dW_r \right]
\end{equation*}
b. Phương trình vi phân tuyến tính với chuyển động Brown nhiều chiều\\
Phương trình vi phân tuyến tính với chuyển động Brown nhiều chiều là phương trình có dạng:
\begin{equation*}
	dX(t)=[\alpha(t)X(t)+f(t)]dt+\sum_{i=1}^{m}[\beta_i(t)X(t)+g_i(t)dW_t^{(i)} \condition{$X(0)=X_0$}
\end{equation*}
Khi đó nghiệm của nó sẽ là:
\begin{equation*}
	\dfrac{X(t)}{V(t)}=X_0+\int_{0}^{t}\dfrac{1}{V(s)}\left(\alpha(s)-\sum_{i=1}^{m}[\beta_i(s)]g_i(s) \right)ds+\int_{0}^{t}\sum_{i=1}^{m}\dfrac{1}{V(s)}g_i(s)dW_{s}^{i}
\end{equation*}
trong đó: $V(t)=\exp\{\int_{0}^{t}[\alpha(s)-\sum_{i=1}^{m}\dfrac{1}{2}\beta_i^2(t)]ds+\int_{0}^{t}\sum_{i=1}^{m}\beta_i(s)dW_s^i \}$.\\
c. Phương trình Black - Scholes\\
Trong toán tài chính, người ta hay nhắc đến phương trình mô tả sự biến động giá $S(t)$ một loại cổ phiếu nào đó theo thời gian:
\begin{equation*}
	dS(t)=\alpha S(t)dt+\beta S(t)dW_t \condition{$S(0)=S_0$}
\end{equation*}
trong đó $\alpha,\beta$ là những hằng số. Đây là phương trình vi phân hình học mà ta đã nói ở phần trên, và nghiệm của nó sẽ là:
\begin{equation*}
	S=(t)=S_0\exp\left[\left(\alpha-\dfrac{\beta^2}{2} \right)t+\beta W_t \right]
\end{equation*}
Chú ý: Khi giải bài toán về định giá quyền chọn mua theo kiểu Châu Âu sẽ dẫn đến một loại phương trình đạo hàm riêng cấp hai khác với phương trình trên nhưng người ta cũng gọi là phương trình Black-Scholesm, đó là phương trình:
\begin{equation*}
	\dfrac{\partial G}{\partial S}+\dfrac{1}{2}\sigma^2S^2\dfrac{\partial^2G}{\partial S^2}+rS\dfrac{\partial G}{\partial S}-rG=0
\end{equation*}
Trong đó: $G=G(S,t)$ là giá quyền chọn tại thời điểm t; giá chứng khoán $S=S(t)$; và lãi suất không rủi ro r.\\
\textbf{3. Ứng dụng trong tài chính}\\
a. Bài toán về chọn danh mục đầu tư rủi ro\\
Trước hết, ta mô hình hoá bài toán về tài sản và phương án đầu tư trên thị trường bằng các công cụ của giải tích ngẫu nhiên. Khi xét đến biến động của dòng tài sản trên thị trường, người ta thường xét đến hai loại tài sản: phi rủi ro (như trái phiếu...) và tài sản rủi ro (như cổ phiếu, bất động sản...)\\
Loại tài sản phi rủi ro với biến động giá $X_0(t)$ được xét qua phương trình vi phân:
\begin{equation*}
	dX_0(t)=\lambda(t,\omega)X_0(t)dt \condition{$X_0(0)=1,t\in[0,T]$}
\end{equation*}
Loại tài sản rủi ro với biến động giá $X_1(t)$ thường được xét qua phương trình vi phân ngẫu nhiên Itô có dạng:
\begin{equation}\label{eq:ud1}
	dX_1(t)=\alpha(t,\omega)X_1(t)dt+\beta(t,\omega)X_1(t)dB_t(\omega) \condition{$X_1(0)>0,t\in[0,T]$}
\end{equation}
trong đó: $\lambda(t,\omega)=\lambda(t);\alpha(t,\omega)=\alpha(t);\beta(t,\omega)=\beta(t)$ là những quá trình ngẫu nhiên thoả điều kiện:
\begin{equation*}
	E\int\limits_{0}^{T}[|\lambda(t)|+\alpha(t)+\beta^2(t)]dt<\infty;
\end{equation*}
Ta ký hiệu: $\tau_0(t)=\tau_0(t,\omega);\tau_1(t)=\tau_1(t)(t,\omega);t\in[0,T];\omega\in \Omega$ là những đơn vị vốn đầu tư cho loại tài sản phi rủi ro và rủi ro tương ứng. Khi đó ta gọi: $\tau(t):=(\tau_0(t),\tau_1(t))$ là một phương án đầu tư (một danh mục đầu tư) với tổng giá trị tài sản tại thời điểm t bằng:
\begin{equation}\label{eq:ud2}
	V^\tau(t)=\tau_0(t)X_0(t)+\tau_1(t)X_1(t)	
\end{equation}
Phương án đầu tư được gọi là tự tài trợ (seft-financing) nếu ta có:
\begin{equation}\label{eq:ud3}
	dVT\tau(t)=\tau_0(t)dX_0(t)+\tau_1(t)dX_1(t)
\end{equation}
Từ giả định: $\tau(t);t\in[0,T]$ là tự tài trợ, từ hệ thức \eqref{eq:ud2} ta sẽ có:
\begin{equation*}
	\tau_0(t)=\dfrac{V^\tau(t)-\tau_1(t)X_1(t)}{X_0(t)}
\end{equation*}
Từ các hệ thức trên suy ra:
\begin{equation*}
	dV^\tau(t)=\lambda(t)(V^\tau(t)-\tau_1(t)X_1(t))dt+\tau_1(t)dX_1(t)
\end{equation*}
Sử dụng \eqref{eq:ud1}, dựa vào phương trình trên ta sẽ thu được:
\begin{equation}\label{eq:ud4}
	dV^\tau(t)=[\lambda(t)V^\tau(t)+(\alpha(t)-\lambda(t))\tau_1(t)X_1(t)]dt +\beta(t)\tau_1(t)X_1(t)dB_t
\end{equation}
Đến đây ta thấy rằng phương trình \eqref{eq:ud4} là một phương trình vi phân ngẫu nhiên tuyến tính mà ta sẽ nói đến phương pháp giải ở mục phía sau.\\
Mặc khác, trong thực tế thường xét đến việc tìm phương án đầu tư $\tau(t);t\in[0,T]$ để thu được tổng giá trị tài sản là: $V^\tau(T)=\text{\AA}$. Đại lượng ngẫu nhiên $\text{\AA}$ tương ứng với quyền tài chính (financial claim), phương án đầu tư như vậy được gọi là phương án đáp ứng (replicating portfolio).\\
Mở rộng vấn đề nêu trên vào trường hợp nhiều chiều, tương ứng với việc có n khối tài sản rủi ro với biến động giá $X_i(t),i=1,2,...,n$, được cho bởi:
\begin{equation}\label{eq:ud5}
	dX_i(t)=X_i(t^-)\left[\beta_i(t)dB_t+\int_{R_0}\gamma_i(t,x)\overline{N}(dt,dx) \right] \condition{$i=1,2,...,n$}
\end{equation}
trong đó $B(t)=(B_1(t),...,B_n(t))^T;t\geq 0$ là quá trình Wiener n-chiều với:
\begin{equation*}
	\overline{N}(dt,dx)=\left(\overline{N}_1(dt,dx_1),...,\overline{N}_n(dt,dx_n)\right)^T \condition{$x\in R_0:=R\backslash\{0\}$}
\end{equation*}
là độ đo ngẫu nhiên Poisson bù (compensated Poisson random measure) và còn:
\begin{equation*}
	\beta_i(t)=(\beta_{i_1}(t),...,\beta_{i_n}(t));\gamma_i(t,x)=(\gamma_{i_1}(t,x),...,\gamma_{i_n}(t,x))
\end{equation*}
là các quá trình ngẫu nhiên thoả điều kiện:
\begin{equation*}
	\sum_{i,j=1}^{n}\int_{0}^{T}\left[\beta_{ij}^2(t)+\int_{R_0}\gamma_{ij}^2(t,x)v(dx) \right]dt<\infty \condition{h.c}
\end{equation*}
Quyền $\text{\AA}$ sẽ có thể đáp ứng nếu tồn tại quá trình ngẫu nhiên:
\begin{equation*}
	\varphi(t)=\left(\varphi_1(t),...\varphi_n(t) \right)^T \condition{$t\geq 0$ sao cho:}\\
\end{equation*}
\begin{equation*}
	\sum_{i=1}^{n}\int_{0}^{T}\varphi_i^2(t)X_i(t^-)\left[\sum_{j=1}^{n}\beta_{ij}^2(t)+\int_{R_0}\gamma_{ij}^2(t,x)v(dx) \right]dt<\infty \condition{h.c}
\end{equation*}
cùng với việc thoả điều kiện $\text{\AA}=E\{\text{\AA}\}+\sum\limits_{i=1}^{n}\int\limits_{0}^{T}\varphi_i(t)dX_i(t)$. Trong trường hợp này, ta nói rằng $\varphi$ là phương án đáp ứng đối với quyền tài chính.\\
b. Giải quyết bài toán đầu tư đáp ứng\\
Để giải quyết bài toán theo hướng đầu tư đáp ứng đã nêu, thường nó dẫn đến phương trình \eqref{eq:ud4} và \eqref{eq:ud5} mà ta xét chứng dưới dạng phương trình vi phân ngẫu nhiên tổng quát sau:\\
\textbf{Định lý:} Cho phương trình vi phân ngẫu nhiên tuyến tính có dạng:
\begin{dmath*}
dX(t)=[\alpha(t)X(t^-)+\mathscr{A}(t)]dt+[\beta(t)X(t^-)+\mathscr{B}(t)]dB_t\\ +\int_{R_0}[\gamma(t,x)+\mathscr{T}(t,x)]\overline{N}(dt,dx) \condition{$X(0)=x_0$}
\end{dmath*}
trong đó: $\alpha(T)=\alpha(t,\omega);\beta(t)=\beta(t,\omega);\gamma(t,x)=\gamma(t,x,\omega);\mathscr{A}(t)=\mathscr{A}(t,\omega);\mathscr{B}(t)=\mathscr{B}(t,\omega);\mathscr{T}(t,x)=\mathscr{T}(t,x,\omega)$ là những quá trình ngẫu nhiên thoả các điều kiện:
\begin{equation*}
\begin{split}
	&\int_{0}^{T}\left[\alpha(t)+\beta^2(t)+\int_{R_0}\gamma^2(t,x)v(dx) \right]dt<\infty\\
	&\int_{0}^{T}\left[\mathscr{A}(t)+\mathscr{B}^2(t)+\int_{R_0}\mathscr{T}^2(t,x)v(dx) \right]dt<\infty
\end{split}
\end{equation*}
Khi đó, ta sẽ có nghiệm của phương trình cho bởi hệ thức:
\begin{dmath*}
\dfrac{X(t)}{X_1(t^-)}=x_0+\int\limits_{0}^{t}X_1^{-1}(s^-)\left[\mathscr{A}(s)-\beta(s)\mathscr{B}(t)-\int\limits_{R_0}\dfrac{\gamma(s,x)\mathscr{T}(s,x)}{1+\gamma(s,x)}v(dx) \right]ds+\int\limits_{0}^{t}\dfrac{\mathscr{B}(s)}{X_1^{-1}(s^-)}dB(s)+\int\limits_{0}^{t}\int\limits_{R_0}\dfrac{\mathscr{T}(t,x)}{X_1(s^-)(1+\gamma(s,x))}\overline{N}(dt,dx).	
\end{dmath*}
trong đó:
\begin{dmath*}
X_1(t^-)=\exp\left\{\int_{0}^{t}\left[\alpha(s)-\dfrac{1}{2}\beta^2(t)+\int_{R_0}[\log(1+\gamma(s,x))-\gamma(s,x)]v(dx) \right]+\int_{0}^{t}\beta(s)dB(s)+\int_{R_0}\int_{0}^{t}\log(1+\gamma(s,x))\overline{N}(ds,dx) \right\}
\end{dmath*}
Để chứng minh định lý này ta dùng phương pháp tách nghiệm đã xét ở phần trước. Ngoài ra, để giải quyết cụ thể hơn các vấn đề về bài toán đầu tư rõ ràng ta phải xét thêm đến những loại quyền chọn cụ thể trên thị trường (quyền chọn Châu Âu, quyền chọn Châu Á...). Việc xem xét và giải chúng trong một vài trường hợp đã cho ta những kết quả nhất định như các kết quả kinh điển của Black-Scholes, Floyd B. Hanson.\\


\textbf{Quyển "Quá trình ngẫu nhiên - Phần 2"}\\
\textbf{Giải phương trình vi phân tuyến tính Itô - Levy}\\
\textbf{Định lý 4.4.1:}\\
Cho phương trình vi phân ngẫu nhiên Itô - Lêvy dạng:
\begin{dmath}\label{eq:4.4.1}
	dX(t)=[\alpha(t,\omega)X(t^-)+A(t,\omega)]dt+[\beta(t,\omega)X(t^-)+B(t,\omega)]dW(t)+\int_{R_0}[\gamma(t,z,\omega)X(t^-)+G(t,z,\omega)]\tilde{N}(dt,dz)
\end{dmath}
với điều kiện ban đầu $X(0)=x_0$. Trong đó:
\begin{equation*}
	\alpha(t,\omega);\beta(t,\omega);A(T,\omega);B(t,\omega);\gamma(t,z,\omega);G(t,z,\omega);t\geq 0;z\in R_0;\omega\in \Omega
\end{equation*}
là những quá trình ngẫu nhiên khả đoán cho trước với
\begin{equation*}
	\gamma(t,z,\omega)>-1;\forall(t,z,\omega)\in[0,\infty)\times R_0\times \Omega
\end{equation*}
và thoả các điều kiện:
\begin{equation*}
	\int_{0}^{t}\left[|\alpha(x,\omega)|-\dfrac{1}{2}\beta^2(s,\omega)+\int_{R_0}\gamma^2(s,z,\omega)v(dz) \right]ds<\infty \condition{h.c}
\end{equation*}
\begin{equation*}
	\int_{0}^{t}\left[|A(s,\omega)|-\dfrac{1}{2}B^2(s,\omega)+\int_{R_0}G^2(s,z,\omega)v(dz) \right]ds<\infty \condition{h.c}
\end{equation*}
Khi đó phương trình trên sẽ có nghiệm là:
\begin{dmath}\label{eq:4.4.2}
X(t)=X_1(t^-)\left\{x_0+\int_0^t\dfrac{1}{X_1(s^-)}\left[A(s,\omega)-\beta(s,\omega)B(s,\omega)-\int_{R_0}\dfrac{\gamma(s,z,\omega)G(s,z,\omega)}{1+\gamma(s,z,\omega)}v(dz)\right]ds+\int_0^t\dfrac{B(s,\omega)}{X_1(s^-)}dW(s)+\int_0^t\int_{R_0}\dfrac{G(s,z,\omega)}{X_1(s^-)(1+\gamma(s,z,\omega))}\tilde{N}(ds,dz) \right\}	
\end{dmath}
trong đó
\begin{dmath}\label{eq:4.4.3}
X_1(t)=\exp\left\{\int_0^t\left[\alpha(s,\omega)-\dfrac{1}{2}\beta^2(s,\omega)+\int_{R_0}[\log(1+\gamma(s,z,\omega))-\gamma(s,z,\omega)]v(dz) \right]ds+\int_0^t\beta(s,\omega)dW(s)+\int_0^t\int_{R_0}\log(1+\gamma(s,t,\omega))\tilde{N}(ds,dz) \right\}	
\end{dmath}
Để chứng minh định lý \eqref{eq:4.4.1} trước hết ta chứng minh bổ đề sau.\\
\textbf{Bổ đề 4.4.2}\\
Cho phương trình vi phân ngẫu nhiên tuyến tính thuần nhất có dạng:
\begin{equation}\label{eq:4.4.4}
	dX_1(t)=X_1(t^-)\left[\alpha(t,\omega)dt+\beta(t,\omega)dW(t)+\int_{R_0}\gamma(t,z,\omega)\tilde{N}(dt,dz) \right]
\end{equation}
với điều kiện ban đầu $X(0)=1$, trong đó: $\alpha(t,\omega);\beta(t,\omega);\gamma(t,z,\omega);t\geq 0;z\in R_0;\omega\in\Omega$ là những quá trình ngẫu nhiên khả đoán cho trước với
\begin{equation*}
	\gamma(t,z,\omega)>-1\condition{h.c $\forall(t,z,\omega)\in[0,\infty)\times R_0\times\Omega$}
\end{equation*}
và thoả điều kiện:
\begin{equation*}
	\int_0^t\left[|\alpha(s,\omega)|-\dfrac{1}{2}\beta^2(s,\omega)+\int_{R_0}\gamma^2(s,z,\omega)v(dz) \right]ds<\infty\condition{h.c}
\end{equation*}
khi đó nghiệm của phương trình \eqref{eq:4.4.4} sẽ cho bởi hệ thức \eqref{eq:4.4.3}.\\
\textbf{Chứng minh bổ đề 4.4.2}\\
Ta xét hàm: $X_1(t)=F(t,H(t));t\geq 0$ với $F(t,x)=e^x$ và $H(t)$ xác định bởi:
\begin{dmath*}
H(t)=\int_0^t\left[\alpha(s,\omega)-\dfrac{1}{2}\beta^2(s,\omega)+\int_{R_0}\left[\log(1+\gamma(s,z,\omega))-\gamma(s,z,\omega)\right]v(dz) \right]	ds+\int_0^t\beta(s,\omega)dW(s)+\int_0^t\int_{R_0}\log(1+\gamma(s,z,\omega))\tilde{N}(ds,dz)
\end{dmath*}
Áp dụng công thức Itô cho $X_1(t)=F(t,H(t))$ ta sẽ thu được:
\begin{dmath*}
	dX_1(t)=e^{H(t)}\left[\left(\alpha(t,\omega)-\dfrac{1}{2}\beta^2(t,\omega)+\int_{R_0}\left[\log(1+\gamma(t,z,\omega))-\gamma(t,z,\omega)\right]v(dz) \right)dt \right]+e^{H(t)}\left[\dfrac{1}{2}\beta^2(t,\omega)dt+\beta(t,\omega)dW(t) \right]+\int_{R_0}e^{H(t)}\left[\gamma(t,z,\omega)-\log(1+\gamma(t,z,\omega)) \right]v(dz)+\int_{R_0}e^{H(t)}\gamma(t,z,\omega)\tilde{N}(dt,dz)=X_1(t^-)\left[\alpha(t,\omega)dt+\beta(t,\omega)dW(t)+\int_{R_0}\gamma(t,z,\omega)\tilde{N}(dt,dz) \right].\blacksquare
\end{dmath*}
\textbf{Chứng minh định lý 4.4.1}\\
Ta tìm nghiệm của phương trình \eqref{eq:4.4.1} bằng phương pháp tách nghiệm, nghĩa là tìm nghiệm của nó dưới dạng tích
\begin{equation}\label{eq:4.4.5}
	X(t)=X_1(t^-).X_2(t^-)
\end{equation}
trong đó:
\begin{itemize}
	\item $X_1(t)$ là nghiệm của phương trình tuyến tính thuần nhất tương ứng, nghĩa là nó là nghiệm của phương trình \eqref{eq:4.4.4} nói trong Bổ đề 4.4.2.
	\item $X_2(t)$ là nghiệm của phương trình:
\begin{equation}\label{eq:4.4.6}
	dX_2(t)=A^*(t,\omega)dt+B^*(t,\omega)dW(t)+\int_{R_0}G^*(t,z,\omega)\tilde{N}(dt,dz)
\end{equation}
với điều kiện $X_2(0)=x_0$, trong đó $A^*(t,\omega);B^*(t,\omega);G^*(t,z,\omega)$ là những hàm ta sẽ xác định sau. 
\end{itemize}
Theo Bổ đề 4.4.2 ta sẽ có nghiệm $X_1(t^-)$ của phương trình \eqref{eq:4.4.4} cho bởi hệ thức \eqref{eq:4.4.3}.\\
Xét hệ quả của định lý về vi phân cho tích các quá trình Itô - Lêvy, khi có hai quá trình Itô - Lêvy một chiều như sau
\begin{equation*}
	dX_i(t)\alpha_i(t,\omega)dt+\beta_i(t,\omega)dW(t)+\int_{R_0}\gamma_i(t,z,\omega)\tilde{N}(dt,dz)\condition{i=1,2,...}
\end{equation*}
ta sẽ có:
\begin{dmath}\label{eq:4.3.3}
d(X_1(t).X_2(t))=X_1(t^-)dX_2(t)+X_2(t^-)dX_1(t)+\beta_1(t,\omega)\beta_1(t,\omega)dt+\int_{R_0}\gamma_1(t,z,\omega)\gamma_1(t,z,\omega)N(dt,dz)	
\end{dmath}
Quay lại bài toán chứng minh, ta áp dụng hệ quả \eqref{eq:4.3.3} cho tích $X(t)=X_1(t^-).X_2(t^-)$ nêu trên ta thu được:
\begin{dmath}\label{eq:4.4.7}
dX(t)=d(X_1(t^-).X_2(t^-))=X_1(t^-)dX_2(t)+X_2(t^-)dX_1(t)+\beta(t,\omega)X_1(t^-)B^*(t,\omega)dt+\int_{R_0}\gamma(t,z,\omega)X_1(t^-)G^*(t,z,\omega)\tilde{N}(dt,dz)=\alpha(t,\omega)X_1(t^-)X_2(t^-)+\beta(t,\omega)X_1(t^-)X_2(t^-)+\int_{R_0}\gamma(t,z,\omega)X_1(t^-)X_2(t^-)\tilde{N}(dt,dz)+X_1(t^-)A^*(t,\omega)dt+X_1(t^-)B^*(t,\omega)dW(t)+X_1(t^-)\int_{R_0}G^*(t,z,\omega)\tilde{N}(dt,dz)+\beta(t,\omega)X_1(t^-)B^*(t,\omega)dt+\int_{R_0}\gamma(t,z,\omega)X_1(t^-)G^*(t,z,\omega)N(dt,dz)	
\end{dmath}
Mặt khác $X(t)$ là nghiệm của phương trình \eqref{eq:4.4.1}, từ đó so sánh giữa \eqref{eq:4.4.1} và \eqref{eq:4.4.7} ta thu được hệ phương trình
\begin{equation*}
	\begin{cases}
		A(t,\omega)=X_1(t^-)\left[A^*(t,\omega)+\beta(t,\omega)B^*(t,\omega)+\int_{R_0}\gamma(t,z,\omega)G(t,z,\omega)v(dz) \right]\\
		B(t,\omega)=X_1(t^-)B^*(t,\omega)\\
		\int_{R_0}G(t,z,\omega)\tilde{N}(dt,dz)=X_1(t^-)\int_{R_0}(1+\gamma(t,z,\omega))G^*(t,z,\omega)\tilde{N}(dt,dz)
	\end{cases}
\end{equation*}
Suy ra:
\begin{equation*}
	\begin{cases}
		A^*(t,\omega)=\dfrac{1}{X_1(t^-)}\left[A(t,\omega)-B(t,\omega)\beta(t,\omega)-\int_{R_0}\dfrac{\gamma(t,z,\omega)G(t,z,\omega)}{1+\gamma(t,z,\omega)}v(dz) \right]\\
		B^*(t,\omega)=\dfrac{B(t,\omega)}{X_1(t^-)}\\
		G^*(t,z,\omega)=\dfrac{G(t,z,\omega)}{X_1(t^-)(1+\gamma(t,z,\omega))}
	\end{cases}
\end{equation*}
Đặt $X_1(t)$ cho bởi hệ thức \eqref{eq:4.4.3}, và các biểu thức của $A^*(t,\omega);B^*(t,\omega);G^*(t,z,\omega)$ đã xác định được vào biểu thức \eqref{eq:4.4.5} ta sẽ có nghiệm \eqref{eq:4.4.2}.$\blacksquare$\\

%Nội dung trọng tâm
\textbf{Phương trình vi phân ngẫu nhiên}\\
Yêu cầu: Dạng phương trình vi phân ngẫu nhiên\\
Nghiệm yếu và nghiệm mạnh\\
Các dạng đặc biệt của phương trình vi phân ngẫu nhiên\\
Phương trình vi phân tuyến tính.\\
\end{document}